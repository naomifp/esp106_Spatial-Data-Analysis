% Options for packages loaded elsewhere
\PassOptionsToPackage{unicode}{hyperref}
\PassOptionsToPackage{hyphens}{url}
%
\documentclass[
]{article}
\usepackage{amsmath,amssymb}
\usepackage{lmodern}
\usepackage{iftex}
\ifPDFTeX
  \usepackage[T1]{fontenc}
  \usepackage[utf8]{inputenc}
  \usepackage{textcomp} % provide euro and other symbols
\else % if luatex or xetex
  \usepackage{unicode-math}
  \defaultfontfeatures{Scale=MatchLowercase}
  \defaultfontfeatures[\rmfamily]{Ligatures=TeX,Scale=1}
\fi
% Use upquote if available, for straight quotes in verbatim environments
\IfFileExists{upquote.sty}{\usepackage{upquote}}{}
\IfFileExists{microtype.sty}{% use microtype if available
  \usepackage[]{microtype}
  \UseMicrotypeSet[protrusion]{basicmath} % disable protrusion for tt fonts
}{}
\makeatletter
\@ifundefined{KOMAClassName}{% if non-KOMA class
  \IfFileExists{parskip.sty}{%
    \usepackage{parskip}
  }{% else
    \setlength{\parindent}{0pt}
    \setlength{\parskip}{6pt plus 2pt minus 1pt}}
}{% if KOMA class
  \KOMAoptions{parskip=half}}
\makeatother
\usepackage{xcolor}
\usepackage[margin=1in]{geometry}
\usepackage{color}
\usepackage{fancyvrb}
\newcommand{\VerbBar}{|}
\newcommand{\VERB}{\Verb[commandchars=\\\{\}]}
\DefineVerbatimEnvironment{Highlighting}{Verbatim}{commandchars=\\\{\}}
% Add ',fontsize=\small' for more characters per line
\usepackage{framed}
\definecolor{shadecolor}{RGB}{248,248,248}
\newenvironment{Shaded}{\begin{snugshade}}{\end{snugshade}}
\newcommand{\AlertTok}[1]{\textcolor[rgb]{0.94,0.16,0.16}{#1}}
\newcommand{\AnnotationTok}[1]{\textcolor[rgb]{0.56,0.35,0.01}{\textbf{\textit{#1}}}}
\newcommand{\AttributeTok}[1]{\textcolor[rgb]{0.77,0.63,0.00}{#1}}
\newcommand{\BaseNTok}[1]{\textcolor[rgb]{0.00,0.00,0.81}{#1}}
\newcommand{\BuiltInTok}[1]{#1}
\newcommand{\CharTok}[1]{\textcolor[rgb]{0.31,0.60,0.02}{#1}}
\newcommand{\CommentTok}[1]{\textcolor[rgb]{0.56,0.35,0.01}{\textit{#1}}}
\newcommand{\CommentVarTok}[1]{\textcolor[rgb]{0.56,0.35,0.01}{\textbf{\textit{#1}}}}
\newcommand{\ConstantTok}[1]{\textcolor[rgb]{0.00,0.00,0.00}{#1}}
\newcommand{\ControlFlowTok}[1]{\textcolor[rgb]{0.13,0.29,0.53}{\textbf{#1}}}
\newcommand{\DataTypeTok}[1]{\textcolor[rgb]{0.13,0.29,0.53}{#1}}
\newcommand{\DecValTok}[1]{\textcolor[rgb]{0.00,0.00,0.81}{#1}}
\newcommand{\DocumentationTok}[1]{\textcolor[rgb]{0.56,0.35,0.01}{\textbf{\textit{#1}}}}
\newcommand{\ErrorTok}[1]{\textcolor[rgb]{0.64,0.00,0.00}{\textbf{#1}}}
\newcommand{\ExtensionTok}[1]{#1}
\newcommand{\FloatTok}[1]{\textcolor[rgb]{0.00,0.00,0.81}{#1}}
\newcommand{\FunctionTok}[1]{\textcolor[rgb]{0.00,0.00,0.00}{#1}}
\newcommand{\ImportTok}[1]{#1}
\newcommand{\InformationTok}[1]{\textcolor[rgb]{0.56,0.35,0.01}{\textbf{\textit{#1}}}}
\newcommand{\KeywordTok}[1]{\textcolor[rgb]{0.13,0.29,0.53}{\textbf{#1}}}
\newcommand{\NormalTok}[1]{#1}
\newcommand{\OperatorTok}[1]{\textcolor[rgb]{0.81,0.36,0.00}{\textbf{#1}}}
\newcommand{\OtherTok}[1]{\textcolor[rgb]{0.56,0.35,0.01}{#1}}
\newcommand{\PreprocessorTok}[1]{\textcolor[rgb]{0.56,0.35,0.01}{\textit{#1}}}
\newcommand{\RegionMarkerTok}[1]{#1}
\newcommand{\SpecialCharTok}[1]{\textcolor[rgb]{0.00,0.00,0.00}{#1}}
\newcommand{\SpecialStringTok}[1]{\textcolor[rgb]{0.31,0.60,0.02}{#1}}
\newcommand{\StringTok}[1]{\textcolor[rgb]{0.31,0.60,0.02}{#1}}
\newcommand{\VariableTok}[1]{\textcolor[rgb]{0.00,0.00,0.00}{#1}}
\newcommand{\VerbatimStringTok}[1]{\textcolor[rgb]{0.31,0.60,0.02}{#1}}
\newcommand{\WarningTok}[1]{\textcolor[rgb]{0.56,0.35,0.01}{\textbf{\textit{#1}}}}
\usepackage{graphicx}
\makeatletter
\def\maxwidth{\ifdim\Gin@nat@width>\linewidth\linewidth\else\Gin@nat@width\fi}
\def\maxheight{\ifdim\Gin@nat@height>\textheight\textheight\else\Gin@nat@height\fi}
\makeatother
% Scale images if necessary, so that they will not overflow the page
% margins by default, and it is still possible to overwrite the defaults
% using explicit options in \includegraphics[width, height, ...]{}
\setkeys{Gin}{width=\maxwidth,height=\maxheight,keepaspectratio}
% Set default figure placement to htbp
\makeatletter
\def\fps@figure{htbp}
\makeatother
\setlength{\emergencystretch}{3em} % prevent overfull lines
\providecommand{\tightlist}{%
  \setlength{\itemsep}{0pt}\setlength{\parskip}{0pt}}
\setcounter{secnumdepth}{-\maxdimen} % remove section numbering
\ifLuaTeX
  \usepackage{selnolig}  % disable illegal ligatures
\fi
\IfFileExists{bookmark.sty}{\usepackage{bookmark}}{\usepackage{hyperref}}
\IfFileExists{xurl.sty}{\usepackage{xurl}}{} % add URL line breaks if available
\urlstyle{same} % disable monospaced font for URLs
\hypersetup{
  pdftitle={Midterm 2},
  hidelinks,
  pdfcreator={LaTeX via pandoc}}

\title{Midterm 2}
\author{}
\date{\vspace{-2.5em}}

\begin{document}
\maketitle

\hypertarget{data}{%
\subsubsection{Data}\label{data}}

For this midterm you need to use two datasets:

``chinalanduse\_MODIS\_2012.nc'' contains four layers with land cover
data for China. The data were derived from MODIS satellite data for the
year 2012. Each layer gives the fraction of the grid cell that has a
specific land cover type: urban (layer 1), cropland (layer 2), grassland
(layer 3) and forest (layer 4).

``ch\_adm.*'' with polygons for the provinces of China.

Q1. Read in the land use data as a SpatRaster get the polygons as a
SpatVector (2 points)

\begin{Shaded}
\begin{Highlighting}[]
\CommentTok{\#load terra package and set working directory}
\FunctionTok{library}\NormalTok{(terra)}
\end{Highlighting}
\end{Shaded}

\begin{verbatim}
## terra 1.7.9
\end{verbatim}

\begin{verbatim}
## 
## Attaching package: 'terra'
\end{verbatim}

\begin{verbatim}
## The following object is masked from 'package:knitr':
## 
##     spin
\end{verbatim}

\begin{Shaded}
\begin{Highlighting}[]
\FunctionTok{setwd}\NormalTok{(}\StringTok{"C:/Users/Benny Panjaitan/Documents/GitHub/esp106{-}Naomi/midterm2"}\NormalTok{)}

\CommentTok{\#store Land Use data as spatial raster and Province data as spatial vector}
\NormalTok{landuse }\OtherTok{\textless{}{-}} \FunctionTok{rast}\NormalTok{(}\StringTok{"chinalanduse\_MODIS\_2012.nc"}\NormalTok{)}
\NormalTok{province }\OtherTok{\textless{}{-}} \FunctionTok{vect}\NormalTok{(}\StringTok{"chn\_adm.shp"}\NormalTok{)}
\end{Highlighting}
\end{Shaded}

\begin{Shaded}
\begin{Highlighting}[]
\CommentTok{\#checking the Province vector data}
\FunctionTok{head}\NormalTok{(province)}
\end{Highlighting}
\end{Shaded}

\begin{verbatim}
##      NAME_1   GID_1 GID_0 COUNTRY VARNAME_1 NL_NAME_1    TYPE_1    ENGTYPE_1
## 1     Anhui CHN.1_1   CHN   China     Ānhuī 安徽|安徽     Shěng     Province
## 2   Beijing CHN.2_1   CHN   China   Běijīng 北京|北京 Zhíxiáshì Municipality
## 3 Chongqing CHN.3_1   CHN   China Chóngqìng 重慶|重庆 Zhíxiáshì Municipality
## 4    Fujian CHN.4_1   CHN   China    Fújiàn      福建     Shěng     Province
## 5     Gansu CHN.5_1   CHN   China     Gānsù 甘肅|甘肃     Shěng     Province
## 6 Guangdong CHN.6_1   CHN   China Guǎngdōng 廣東|广东     Shěng     Province
##   CC_1 HASC_1 ISO_1
## 1 <NA>  CN.AH CN-AH
## 2 <NA>  CN.BJ CN-BJ
## 3 <NA>  CN.CQ CN-CQ
## 4 <NA>  CN.FJ CN-FJ
## 5 <NA>  CN.GS CN-GS
## 6 <NA>  CN.GD CN-GD
\end{verbatim}

Q2a. Crop the land use SpatRaster to the same extent as the SpatVector
of Chinese provinces (1 point), and set all grid cells outside of China
to \texttt{NA}

\begin{Shaded}
\begin{Highlighting}[]
\CommentTok{\#cropping the Land Use data according to the Province spatial range}
\NormalTok{e }\OtherTok{\textless{}{-}} \FunctionTok{ext}\NormalTok{(province)}
\NormalTok{chlu }\OtherTok{\textless{}{-}} \FunctionTok{crop}\NormalTok{(landuse, e, }\AttributeTok{snap=}\StringTok{"in"}\NormalTok{, }\AttributeTok{extend=}\ConstantTok{FALSE}\NormalTok{)}
\end{Highlighting}
\end{Shaded}

Q2b. Rename the layers in the SpatRaster so they provide information
about what data is in each of the 4 layers (2 points)

\begin{Shaded}
\begin{Highlighting}[]
\CommentTok{\#renaming Land Use layers}
\FunctionTok{names}\NormalTok{(chlu) }\OtherTok{\textless{}{-}} \FunctionTok{c}\NormalTok{(}\StringTok{"Urban"}\NormalTok{, }\StringTok{"Cropland"}\NormalTok{, }\StringTok{"Grassland"}\NormalTok{, }\StringTok{"Forest"}\NormalTok{)}

\CommentTok{\#plotting to check}
\FunctionTok{plot}\NormalTok{ (chlu)}
\end{Highlighting}
\end{Shaded}

\includegraphics{Midterm-2_files/figure-latex/unnamed-chunk-4-1.pdf}

Q3. Make a figure showing each SpatRaster layer on one of the panels and
overlay the polygons of the Chinese provinces. Title each panel with the
type of land use it shows. (4 points)

\begin{Shaded}
\begin{Highlighting}[]
\CommentTok{\#plotting the Land Use classification by China provinces map}
\FunctionTok{plot}\NormalTok{(chlu, }\AttributeTok{fun=}\ControlFlowTok{function}\NormalTok{() }\FunctionTok{lines}\NormalTok{(province))}
\end{Highlighting}
\end{Shaded}

\includegraphics{Midterm-2_files/figure-latex/unnamed-chunk-5-1.pdf}

Q4a. Use \texttt{terra::extract} to find the fraction of each province
in each of the four land use classes. {[}For this question you can
assume all the grid cells have the same size{]} (3 points)

\begin{Shaded}
\begin{Highlighting}[]
\CommentTok{\#calculating the fraction of each land use type in every provinces}
\NormalTok{china.df }\OtherTok{\textless{}{-}} \FunctionTok{extract}\NormalTok{(chlu, province, }\AttributeTok{fun=}\StringTok{"mean"}\NormalTok{)}
\NormalTok{china.df}\SpecialCharTok{$}\NormalTok{ID }\OtherTok{\textless{}{-}}\NormalTok{ province}\SpecialCharTok{$}\NormalTok{NAME\_1}
\NormalTok{china.df}
\end{Highlighting}
\end{Shaded}

\begin{verbatim}
##                ID        Urban    Cropland   Grassland       Forest
## 1           Anhui 1.485168e-02 0.646405733 0.003847900 1.387219e-03
## 2         Beijing 1.153714e-01 0.299729895 0.144125230 3.093923e-03
## 3       Chongqing 6.521529e-03 0.203184739 0.001481789 5.244315e-03
## 4          Fujian 2.600561e-02 0.061434232 0.004364550 3.092684e-05
## 5           Gansu 3.996819e-03 0.087680624 0.400894396 3.527299e-03
## 6       Guangdong 4.223879e-02 0.119822469 0.009257352 1.260333e-04
## 7         Guangxi 1.071244e-02 0.051608185 0.002281014 6.499771e-05
## 8         Guizhou 3.764910e-03 0.153545678 0.007175318 3.731592e-04
## 9          Hainan 7.707157e-03 0.086235658 0.002913859 4.523766e-03
## 10          Hebei 3.362038e-02 0.483198482 0.306454951 7.317166e-03
## 11   Heilongjiang 7.829202e-03 0.389870519 0.033178216 7.253880e-02
## 12          Henan 3.850683e-02 0.782530859 0.009605914 2.720180e-02
## 13      Hong Kong 0.000000e+00 0.000000000 0.000000000 0.000000e+00
## 14          Hubei 1.131149e-02 0.390618615 0.004631285 1.463583e-02
## 15          Hunan 9.341245e-03 0.185297511 0.003591011 1.076864e-04
## 16        Jiangsu 3.326057e-02 0.768018677 0.014393223 2.106749e-04
## 17        Jiangxi 7.923291e-03 0.222333333 0.005902371 5.718270e-05
## 18          Jilin 1.181648e-02 0.385392242 0.097113314 1.270092e-01
## 19       Liaoning 2.529342e-02 0.519516122 0.051588569 6.504468e-02
## 20          Macau 0.000000e+00 0.033333334 0.026666666 0.000000e+00
## 21     Nei Mongol 2.629228e-03 0.081413187 0.525555413 1.214391e-03
## 22    Ningxia Hui 7.515666e-03 0.150782292 0.682385283 1.455427e-04
## 23        Qinghai 1.097182e-03 0.002437022 0.689000417 1.850525e-05
## 24        Shaanxi 8.537541e-03 0.244829928 0.273652750 4.651758e-02
## 25       Shandong 4.267022e-02 0.851962730 0.028954643 1.324187e-04
## 26       Shanghai 2.146548e-01 0.553956132 0.024272948 1.949634e-04
## 27         Shanxi 2.012471e-02 0.360992363 0.402905064 2.882390e-03
## 28        Sichuan 3.773638e-03 0.154516600 0.386690523 1.759347e-03
## 29        Tianjin 8.374076e-02 0.669960851 0.080365376 2.435842e-04
## 30 Xinjiang Uygur 1.696892e-03 0.039337617 0.247525979 1.484126e-05
## 31         Xizang 6.523447e-05 0.002027606 0.617138677 1.069450e-04
## 32         Yunnan 7.839736e-03 0.085950516 0.065110655 1.404948e-04
## 33       Zhejiang 3.024358e-02 0.193322905 0.007327374 2.703911e-04
\end{verbatim}

Q4b. Describe the potential problem with the area assumption made in 4a.
How might it affect the calculation in that step? What could we do if we
didn't want to make that assumption? (You don't have to do it, just
describe in theory) (2 points)

\textbf{Answer:} Because all grid cells are assumed to have same size,
they may be overlapping in each province's sizes. This may make the
fractions not accurate. The alternative way other than making that
assumption is by computing the size of grid cells by terra::area
function.

Q4c. Sum up the fractions in the four land cover classes for each
province and plot these as a histogram. (2 points)

\begin{Shaded}
\begin{Highlighting}[]
\CommentTok{\#calculating the total of land coverage specified by 2012 MODIS satellite data }
\NormalTok{china.df}\SpecialCharTok{$}\NormalTok{Total }\OtherTok{\textless{}{-}} \FunctionTok{rowSums}\NormalTok{(china.df[,}\DecValTok{2}\SpecialCharTok{:}\DecValTok{5}\NormalTok{])}

\CommentTok{\#plotting the frequencies of land coverage in China}
\FunctionTok{hist}\NormalTok{(china.df}\SpecialCharTok{$}\NormalTok{Total, }\AttributeTok{main=}\StringTok{"Land Coverage in China"}\NormalTok{, }\AttributeTok{xlab=}\StringTok{"Fraction of Land Coverage"}\NormalTok{)}
\end{Highlighting}
\end{Shaded}

\includegraphics{Midterm-2_files/figure-latex/unnamed-chunk-7-1.pdf}

Q5. Add a new variable called ``other'' to the data.frame created with
terra::extract. This variable should represent the fraction of all other
land cover classes. Assign it the appropriate values. (2 points)

\begin{Shaded}
\begin{Highlighting}[]
\CommentTok{\#Calculating the "Other" type of land than 4 land use type defined}
\NormalTok{china.df}\SpecialCharTok{$}\NormalTok{Other }\OtherTok{\textless{}{-}} \DecValTok{1}\SpecialCharTok{{-}}\NormalTok{china.df}\SpecialCharTok{$}\NormalTok{Total}
\NormalTok{china.df}
\end{Highlighting}
\end{Shaded}

\begin{verbatim}
##                ID        Urban    Cropland   Grassland       Forest      Total
## 1           Anhui 1.485168e-02 0.646405733 0.003847900 1.387219e-03 0.66649253
## 2         Beijing 1.153714e-01 0.299729895 0.144125230 3.093923e-03 0.56232044
## 3       Chongqing 6.521529e-03 0.203184739 0.001481789 5.244315e-03 0.21643237
## 4          Fujian 2.600561e-02 0.061434232 0.004364550 3.092684e-05 0.09183531
## 5           Gansu 3.996819e-03 0.087680624 0.400894396 3.527299e-03 0.49609914
## 6       Guangdong 4.223879e-02 0.119822469 0.009257352 1.260333e-04 0.17144464
## 7         Guangxi 1.071244e-02 0.051608185 0.002281014 6.499771e-05 0.06466663
## 8         Guizhou 3.764910e-03 0.153545678 0.007175318 3.731592e-04 0.16485907
## 9          Hainan 7.707157e-03 0.086235658 0.002913859 4.523766e-03 0.10138044
## 10          Hebei 3.362038e-02 0.483198482 0.306454951 7.317166e-03 0.83059097
## 11   Heilongjiang 7.829202e-03 0.389870519 0.033178216 7.253880e-02 0.50341673
## 12          Henan 3.850683e-02 0.782530859 0.009605914 2.720180e-02 0.85784540
## 13      Hong Kong 0.000000e+00 0.000000000 0.000000000 0.000000e+00 0.00000000
## 14          Hubei 1.131149e-02 0.390618615 0.004631285 1.463583e-02 0.42119722
## 15          Hunan 9.341245e-03 0.185297511 0.003591011 1.076864e-04 0.19833745
## 16        Jiangsu 3.326057e-02 0.768018677 0.014393223 2.106749e-04 0.81588315
## 17        Jiangxi 7.923291e-03 0.222333333 0.005902371 5.718270e-05 0.23621618
## 18          Jilin 1.181648e-02 0.385392242 0.097113314 1.270092e-01 0.62133120
## 19       Liaoning 2.529342e-02 0.519516122 0.051588569 6.504468e-02 0.66144279
## 20          Macau 0.000000e+00 0.033333334 0.026666666 0.000000e+00 0.06000000
## 21     Nei Mongol 2.629228e-03 0.081413187 0.525555413 1.214391e-03 0.61081222
## 22    Ningxia Hui 7.515666e-03 0.150782292 0.682385283 1.455427e-04 0.84082878
## 23        Qinghai 1.097182e-03 0.002437022 0.689000417 1.850525e-05 0.69255313
## 24        Shaanxi 8.537541e-03 0.244829928 0.273652750 4.651758e-02 0.57353780
## 25       Shandong 4.267022e-02 0.851962730 0.028954643 1.324187e-04 0.92372001
## 26       Shanghai 2.146548e-01 0.553956132 0.024272948 1.949634e-04 0.79307880
## 27         Shanxi 2.012471e-02 0.360992363 0.402905064 2.882390e-03 0.78690453
## 28        Sichuan 3.773638e-03 0.154516600 0.386690523 1.759347e-03 0.54674011
## 29        Tianjin 8.374076e-02 0.669960851 0.080365376 2.435842e-04 0.83431057
## 30 Xinjiang Uygur 1.696892e-03 0.039337617 0.247525979 1.484126e-05 0.28857533
## 31         Xizang 6.523447e-05 0.002027606 0.617138677 1.069450e-04 0.61933846
## 32         Yunnan 7.839736e-03 0.085950516 0.065110655 1.404948e-04 0.15904140
## 33       Zhejiang 3.024358e-02 0.193322905 0.007327374 2.703911e-04 0.23116425
##         Other
## 1  0.33350747
## 2  0.43767956
## 3  0.78356763
## 4  0.90816469
## 5  0.50390086
## 6  0.82855536
## 7  0.93533337
## 8  0.83514093
## 9  0.89861956
## 10 0.16940903
## 11 0.49658327
## 12 0.14215460
## 13 1.00000000
## 14 0.57880278
## 15 0.80166255
## 16 0.18411685
## 17 0.76378382
## 18 0.37866880
## 19 0.33855721
## 20 0.94000000
## 21 0.38918778
## 22 0.15917122
## 23 0.30744687
## 24 0.42646220
## 25 0.07627999
## 26 0.20692120
## 27 0.21309547
## 28 0.45325989
## 29 0.16568943
## 30 0.71142467
## 31 0.38066154
## 32 0.84095860
## 33 0.76883575
\end{verbatim}

Q6. Make barplots showing the breakdown of urban, cropland, grassland,
forest, and other for each province. The barplots should be ``stacked''
(a single bar for each province, showing land cover with a color) and
``horizontal'' (province names on the vertical axis).

Q6a) Use graphics::barplot, make sure to include a legend. (4 points)

First, I prepare the matrix for barplotting using R base

\begin{Shaded}
\begin{Highlighting}[]
\CommentTok{\#removing the "Total" column and renaming the columns\textquotesingle{} name}
\NormalTok{china.df }\OtherTok{\textless{}{-}} \FunctionTok{subset}\NormalTok{(china.df, }\AttributeTok{select=}\SpecialCharTok{{-}}\FunctionTok{c}\NormalTok{(Total))}
\FunctionTok{colnames}\NormalTok{(china.df) }\OtherTok{\textless{}{-}} \FunctionTok{c}\NormalTok{(}\StringTok{"Province"}\NormalTok{, }\StringTok{"Urban"}\NormalTok{, }\StringTok{"Cropland"}\NormalTok{, }\StringTok{"Grassland"}\NormalTok{, }\StringTok{"Forest"}\NormalTok{, }\StringTok{"Other"}\NormalTok{)}

\CommentTok{\#convert the data frame into matrix for barplotting and renaming the first column\textquotesingle{}s name}
\NormalTok{china.mat }\OtherTok{\textless{}{-}} \FunctionTok{as.matrix}\NormalTok{(}\FunctionTok{subset}\NormalTok{(china.df, }\AttributeTok{select=}\SpecialCharTok{{-}}\FunctionTok{c}\NormalTok{(Province)))}
\FunctionTok{rownames}\NormalTok{(china.mat) }\OtherTok{\textless{}{-}}\NormalTok{ china.df}\SpecialCharTok{$}\NormalTok{Province}

\CommentTok{\#checking the matrix}
\NormalTok{china.mat}
\end{Highlighting}
\end{Shaded}

\begin{verbatim}
##                       Urban    Cropland   Grassland       Forest      Other
## Anhui          1.485168e-02 0.646405733 0.003847900 1.387219e-03 0.33350747
## Beijing        1.153714e-01 0.299729895 0.144125230 3.093923e-03 0.43767956
## Chongqing      6.521529e-03 0.203184739 0.001481789 5.244315e-03 0.78356763
## Fujian         2.600561e-02 0.061434232 0.004364550 3.092684e-05 0.90816469
## Gansu          3.996819e-03 0.087680624 0.400894396 3.527299e-03 0.50390086
## Guangdong      4.223879e-02 0.119822469 0.009257352 1.260333e-04 0.82855536
## Guangxi        1.071244e-02 0.051608185 0.002281014 6.499771e-05 0.93533337
## Guizhou        3.764910e-03 0.153545678 0.007175318 3.731592e-04 0.83514093
## Hainan         7.707157e-03 0.086235658 0.002913859 4.523766e-03 0.89861956
## Hebei          3.362038e-02 0.483198482 0.306454951 7.317166e-03 0.16940903
## Heilongjiang   7.829202e-03 0.389870519 0.033178216 7.253880e-02 0.49658327
## Henan          3.850683e-02 0.782530859 0.009605914 2.720180e-02 0.14215460
## Hong Kong      0.000000e+00 0.000000000 0.000000000 0.000000e+00 1.00000000
## Hubei          1.131149e-02 0.390618615 0.004631285 1.463583e-02 0.57880278
## Hunan          9.341245e-03 0.185297511 0.003591011 1.076864e-04 0.80166255
## Jiangsu        3.326057e-02 0.768018677 0.014393223 2.106749e-04 0.18411685
## Jiangxi        7.923291e-03 0.222333333 0.005902371 5.718270e-05 0.76378382
## Jilin          1.181648e-02 0.385392242 0.097113314 1.270092e-01 0.37866880
## Liaoning       2.529342e-02 0.519516122 0.051588569 6.504468e-02 0.33855721
## Macau          0.000000e+00 0.033333334 0.026666666 0.000000e+00 0.94000000
## Nei Mongol     2.629228e-03 0.081413187 0.525555413 1.214391e-03 0.38918778
## Ningxia Hui    7.515666e-03 0.150782292 0.682385283 1.455427e-04 0.15917122
## Qinghai        1.097182e-03 0.002437022 0.689000417 1.850525e-05 0.30744687
## Shaanxi        8.537541e-03 0.244829928 0.273652750 4.651758e-02 0.42646220
## Shandong       4.267022e-02 0.851962730 0.028954643 1.324187e-04 0.07627999
## Shanghai       2.146548e-01 0.553956132 0.024272948 1.949634e-04 0.20692120
## Shanxi         2.012471e-02 0.360992363 0.402905064 2.882390e-03 0.21309547
## Sichuan        3.773638e-03 0.154516600 0.386690523 1.759347e-03 0.45325989
## Tianjin        8.374076e-02 0.669960851 0.080365376 2.435842e-04 0.16568943
## Xinjiang Uygur 1.696892e-03 0.039337617 0.247525979 1.484126e-05 0.71142467
## Xizang         6.523447e-05 0.002027606 0.617138677 1.069450e-04 0.38066154
## Yunnan         7.839736e-03 0.085950516 0.065110655 1.404948e-04 0.84095860
## Zhejiang       3.024358e-02 0.193322905 0.007327374 2.703911e-04 0.76883575
\end{verbatim}

Then, I plot the matrix into barplot.

\begin{Shaded}
\begin{Highlighting}[]
\CommentTok{\#barplotting using base R}
\FunctionTok{library}\NormalTok{(colorspace)}
\end{Highlighting}
\end{Shaded}

\begin{verbatim}
## 
## Attaching package: 'colorspace'
\end{verbatim}

\begin{verbatim}
## The following object is masked from 'package:terra':
## 
##     RGB
\end{verbatim}

\begin{Shaded}
\begin{Highlighting}[]
\FunctionTok{par}\NormalTok{(}\AttributeTok{mar=}\FunctionTok{c}\NormalTok{(}\DecValTok{5}\NormalTok{, }\DecValTok{8}\NormalTok{, }\DecValTok{4}\NormalTok{, }\DecValTok{8}\NormalTok{), }\AttributeTok{xpd=}\ConstantTok{TRUE}\NormalTok{)}
\FunctionTok{barplot}\NormalTok{(}\FunctionTok{t}\NormalTok{(china.mat), }
        \AttributeTok{main=}\StringTok{"Land Use for each Province in China"}\NormalTok{, }
        \AttributeTok{xlab=}\StringTok{"Fraction of each Land Cover"}\NormalTok{,}
        \AttributeTok{cex.names=}\FloatTok{0.4}\NormalTok{,}
        \AttributeTok{legend.text=}\NormalTok{T,}
        \AttributeTok{args.legend =} \FunctionTok{list}\NormalTok{(}\AttributeTok{x =} \StringTok{"topright"}\NormalTok{, }\AttributeTok{inset =} \FunctionTok{c}\NormalTok{(}\SpecialCharTok{{-}}\FloatTok{0.35}\NormalTok{, }\DecValTok{0}\NormalTok{)),}
        \AttributeTok{beside=}\NormalTok{F, }
        \AttributeTok{horiz=}\NormalTok{T, }
        \AttributeTok{las=}\DecValTok{1}\NormalTok{, }
        \AttributeTok{col=}\FunctionTok{terrain\_hcl}\NormalTok{(}\DecValTok{5}\NormalTok{)}
\NormalTok{        )}
\end{Highlighting}
\end{Shaded}

\includegraphics{Midterm-2_files/figure-latex/unnamed-chunk-10-1.pdf}

Q6b) Use ggplot. (4 points)

\begin{Shaded}
\begin{Highlighting}[]
\NormalTok{ch.df }\OtherTok{\textless{}{-}}\NormalTok{ tidyr}\SpecialCharTok{::}\FunctionTok{pivot\_longer}\NormalTok{(china.df, }\AttributeTok{cols=}\DecValTok{2}\SpecialCharTok{:}\DecValTok{6}\NormalTok{, }\AttributeTok{values\_to=}\StringTok{"Fraction"}\NormalTok{, }\AttributeTok{names\_to=}\StringTok{"Land\_Use"}\NormalTok{)}

\FunctionTok{library}\NormalTok{(ggplot2)}
\FunctionTok{ggplot}\NormalTok{(ch.df, }\FunctionTok{aes}\NormalTok{(}\AttributeTok{x=}\NormalTok{Province, }\AttributeTok{y=}\NormalTok{Fraction, }\AttributeTok{fill=}\NormalTok{Land\_Use)) }\SpecialCharTok{+}
  \FunctionTok{geom\_bar}\NormalTok{(}\AttributeTok{position=}\StringTok{"stack"}\NormalTok{, }\AttributeTok{stat=}\StringTok{"identity"}\NormalTok{)  }\SpecialCharTok{+}
  \FunctionTok{coord\_flip}\NormalTok{() }\SpecialCharTok{+}
  \FunctionTok{labs}\NormalTok{(}\AttributeTok{title=}\StringTok{"Distribution of Land Use in China"}\NormalTok{)}
\end{Highlighting}
\end{Shaded}

\includegraphics{Midterm-2_files/figure-latex/unnamed-chunk-11-1.pdf}

Q7. Upload your R markdown file, and your knitted output to Canvas. Push
the R markdown file to your Github repository. (2 points)

\end{document}
